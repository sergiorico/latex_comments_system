%========================= LaTeX Unified Comment System =========================
% File: comments.tex  —  inline notes + author comments (ACM/IEEE/etc friendly)

% 1) Packages
\RequirePackage{xcolor}
\RequirePackage{xspace}
\RequirePackage{fontawesome5}
\RequirePackage{etoolbox}

% 2) Global toggle
\newif\ifshowcomments
\showcommentstrue            % set \showcommentsfalse for camera-ready

% 3) Colors
\definecolor{authoronecolor}{RGB}{0,102,204}
\definecolor{authortwocolor}{RGB}{204,0,0}
\definecolor{authorthreecolor}{RGB}{0,153,0}
\definecolor{authorfourcolor}{RGB}{128,0,128}

\definecolor{hintcite}{RGB}{0,102,204}
\definecolor{hintlang}{RGB}{0,153,0}
\definecolor{hintedit}{RGB}{204,0,0}
\definecolor{hintwarn}{RGB}{204,102,0}
\definecolor{hintcheck}{RGB}{90,90,90}
\definecolor{hintfig}{RGB}{128,0,128}
\definecolor{hinttodo}{RGB}{255,140,0}
\definecolor{hintreview}{RGB}{138,43,226}

% Map logical types to actual color names (avoid @ in color names)
\colorlet{cmtcol-cite}{hintcite}
\colorlet{cmtcol-lang}{hintlang}
\colorlet{cmtcol-edit}{hintedit}
\colorlet{cmtcol-warn}{hintwarn}
\colorlet{cmtcol-check}{hintcheck}
\colorlet{cmtcol-fig}{hintfig}
\colorlet{cmtcol-todo}{hinttodo}
\colorlet{cmtcol-review}{hintreview}

% Helper: build the color name string from the type
\newcommand{\cmtcolor}[1]{cmtcol-#1}

% Helper: map comment types to FontAwesome icons
\newcommand{\cmticon}[1]{%
  \if\relax\detokenize{#1}\relax
    \faIcon{exclamation-triangle} % default for empty
  \else
    \ifstrequal{#1}{cite}{\faIcon{quote-left}}{%
    \ifstrequal{#1}{edit}{\faIcon{pencil-alt}}{%
    \ifstrequal{#1}{lang}{\faIcon{language}}{%
    \ifstrequal{#1}{warn}{\faIcon{exclamation-triangle}}{%
    \ifstrequal{#1}{check}{\faIcon{check-circle}}{%
    \ifstrequal{#1}{fig}{\faIcon{image}}{%
    \ifstrequal{#1}{todo}{\faIcon{list-ul}}{%
    \ifstrequal{#1}{review}{\faIcon{eye}}{%
      \faIcon{exclamation-triangle}% default for unknown
    }}}}}}}}%
  \fi
}

% 4) Names and counter
\newcommand{\AuthorOneName}{Author 1}
\newcommand{\AuthorTwoName}{Author 2}
\newcommand{\AuthorThreeName}{Author 3}
\newcommand{\AuthorFourName}{Author X}
\newcounter{commentcounter}

% 5) Header helper
\newcommand{\printcommentheader}[2]{%
  {\color{#1}\textbf{[#2 \thecommentcounter]}}%
}

% 6) Author comment macros (inline or block)
\ifshowcomments
  \long\def\authorone#1{%
    \ifhmode
      \printcommentheader{authoronecolor}{\AuthorOneName}%
      {\color{authoronecolor} #1}\space\stepcounter{commentcounter}%
    \else
      \printcommentheader{authoronecolor}{\AuthorOneName}\par\smallskip
      {\color{authoronecolor}#1}\par\stepcounter{commentcounter}%
    \fi
  }
  \long\def\authortwo#1{%
    \ifhmode
      \printcommentheader{authortwocolor}{\AuthorTwoName}%
      {\color{authortwocolor} #1}\space\stepcounter{commentcounter}%
    \else
      \printcommentheader{authortwocolor}{\AuthorTwoName}\par\smallskip
      {\color{authortwocolor}#1}\par\stepcounter{commentcounter}%
    \fi
  }
  \long\def\authorthree#1{%
    \ifhmode
      \printcommentheader{authorthreecolor}{\AuthorThreeName}%
      {\color{authorthreecolor} #1}\space\stepcounter{commentcounter}%
    \else
      \printcommentheader{authorthreecolor}{\AuthorThreeName}\par\smallskip
      {\color{authorthreecolor}#1}\par\stepcounter{commentcounter}%
    \fi
  }
  \long\def\authorfour#1{%
    \ifhmode
      \printcommentheader{authorfourcolor}{\AuthorFourName}%
      {\color{authorfourcolor} #1}\space\stepcounter{commentcounter}%
    \else
      \printcommentheader{authorfourcolor}{\AuthorFourName}\par\smallskip
      {\color{authorfourcolor}#1}\par\stepcounter{commentcounter}%
    \fi
  }
\else
  \newcommand{\authorone}[1]{}
  \newcommand{\authortwo}[1]{}
  \newcommand{\authorthree}[1]{}
  \newcommand{\authorfour}[1]{}
\fi

% 7) Aliases/utilities
\newcommand{\aone}{\authorone}
\newcommand{\atwo}{\authortwo}
\newcommand{\athree}{\authorthree}
\newcommand{\ax}{\authorfour}

% Natural author name commands - predefined based on common names
% These need to be defined in the document after setting author names
\newcommand{\createauthorcommand}[2]{%
  % #1 = command name (without backslash), #2 = author number (1,2,3,4)
  \ifnum#2=1\relax
    \expandafter\newcommand\csname#1\endcsname{\authorone}%
  \else\ifnum#2=2\relax
    \expandafter\newcommand\csname#1\endcsname{\authortwo}%
  \else\ifnum#2=3\relax
    \expandafter\newcommand\csname#1\endcsname{\authorthree}%
  \else\ifnum#2=4\relax
    \expandafter\newcommand\csname#1\endcsname{\authorfour}%
  \fi\fi\fi\fi
}

% Utility commands
\newcommand{\resetcomments}{\setcounter{commentcounter}{0}}
\newcommand{\commentcount}{\textit{[Total comments: \thecommentcounter]}}

% Show/hide comments commands for easy toggling
\newcommand{\showcomments}{\showcommentstrue}
\newcommand{\hidecomments}{\showcommentsfalse}

% 8) Inline notes (public API = \addnote)
%    Usage: \addnote[<type>]{<text>}  (type defaults to 'cite')
%    Types: cite, edit, lang, warn, check, fig, todo, review
\ifshowcomments
  \DeclareRobustCommand{\addnote}[2][cite]{%
    \textcolor{\cmtcolor{#1}}{\cmticon{#1}~\textbf{[#1]}~#2}\xspace
  }
\else
  \DeclareRobustCommand{\addnote}[2][]{}
\fi
%==============================================================================%
